\documentclass[english]{tutkielma}

\title{The example of the thesis style}
\foreigntitle{Tutkielma-tyylin esimerkki}
\author{FirstName}{SecondName}{LastName}
\location{Jyväskylä}
\university{University of Jyväskylä}
\foreignuniversity{Jyväskylän yliopisto}
\department{Department of Computer Science and Information Systems}
\subject{Information Systems Science}
\type{Example}
\foreigntype{Esimerkki}
\keywords{\LaTeX, layout}
\foreignkeywords{\LaTeX, taitto}

%\fixdate{13}{7}{2007}

\usepackage{texnames}
\usepackage[pdftex]{graphicx}

\begin{document}

\maketitle

\begin{abstract}

\noindent This is an example about tutkielma"-style.
In addition to that previous sentence, this abstract should also have
an another one so that we can see, how hyphenation and other
typography works.

\end{abstract}

\begin{foreignabstract}

\noindent Tämä on esimerkki tutkielma"-tyylin käytöstä.
Tässä abstraktissa on hyvä olla vielä toinenkin virke,
jotta kävisi ilmi, kuinka kappale rivittyy.


\end{foreignabstract}

\chapter*{Foreword}
\thispagestyle{empty}

\noindent This is a Foreword and it is quite short. Following pages use Finnish example chapters and figures. This English example document only shows the basic usage of tutkielma"-style in with English documents.

\chapter*{Figures}
\listoffigures
\thispagestyle{empty}

% Hack two chapters on the same page
\begingroup
\renewcommand{\cleardoublepage}{}
\renewcommand{\clearpage}{}
\par\addvspace{26pt}

\chapter*{Tables}
\listoftables
\thispagestyle{empty}

\endgroup

\tableofcontents

\include{luvut/esimerkkiluku}

\bibliography{esimerkki}

\appendix

\include{luvut/esimerkkiliite}

\end{document}
